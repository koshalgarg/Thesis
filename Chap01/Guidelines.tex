%Formatting Guidelines for Writing Dissertation.
\chapter{Introduction}

Outlier detection is used in variety of applications, such as fraud detection for credit cards, insurance, or health care, intrusion detection
for cyber-security, fault detection in safety critical systems, and military surveillance
for enemy activities. However, outlier detection on streaming data is particularly
challenging, since the volume of data to be analyzed is
effectively unbounded and cannot be stored indefinitely in
memory for processing [4]. Data streams are generated at a high data rate and hence the computation speed and efficiency of algorithm has to be high. An outlier detection system in wireless sensor
networks must work with the limited memory in each
sensor node in order to detect rare events in near real time. In the
case of data streams, where the number of data points is
unbounded and can arrive at a high rate, keeping all data
points is impossible. Simply deleting some of the points does not help because it may affect the accuracy and detection efficiency of upcoming points. Deleting previous points can cause two problems: i) Deleting the previous data points decreases the detection accuracy of local outlier factor for new data points, ii)We can not differentiate between past events and new events. 


Anomaly detection refers to the problem of finding patterns in data that do not conform
to expected behavior. These nonconforming patterns are often referred to as anomalies,
outliers, discordant observations, exceptions, aberrations, surprises, peculiarities, or contaminants in different application domains. [1]. Anomalies are patterns in data that do not conform to a well defined notion of normal
behavior.



\section{Thesis Organization}

In this project we are using relative density based anomaly detection technique, LOF to calculate the anomaly score of data points. The objective of this project is to use LOF technique to find outliers in streaming data. e work done in this project can be summed up as
follows:


	




