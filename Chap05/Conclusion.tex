\chapter{Conclusion and Future Scope}\label{Conclusion and Future Scope}
Anomaly detection has many applications in various sectors. However anomaly detection in
streaming data is preferred over static data because its not possible to have and store all the data
in memory. Continuous generation of data and the need to identify the nature of point as fast as
possible motivated us to research on this topic.
Selection of points which are to be summarized and deleted is very crucial because that determines
the accuracy of LOF detection. In MiLOF they select the points which come early in the data stream
assuming that new points are more crucial for anomaly detection. Since there is no selection criteria
for these points many crucial points are summarized and deleted. What we have suggested in MiLOF
with RKNN is that we store number of reverse K nearest neighbors(RKNN) of all the points and
when memory limit is reached we select the points with high RKNN value. High RKNN value here
means the point is surrounded by sufficient number of points which ensures the point to be normal.
By doing so we are keeping the candidate outliers for further computations. From the results plots
it is quite evident that MiLOF with RKNN provides better results then normal MiLOF.

\section{Future Scope}

The outlier detection technique for streaming data discussed in Chapter 3 can be explored further and better summarization techniques other than $c$-means clustering can be used. Deletion of data points with high RKNN values increases the accuracy and hence can be used in many domains like


	\begin{itemize}
		
		\item Fraud detection for credit cards
		\item Intrusion detection
		\item Traffic anomaly detection 
		\item Medical and public health anomaly detection
		\item Sensor networks
	\end{itemize}


